% abtex2-modelo-artigo.tex, v-1.9.2 laurocesar
% Copyright 2012-2014 by abnTeX2 group at http://abntex2.googlecode.com/ 
%

% ------------------------------------------------------------------------
% ------------------------------------------------------------------------
% abnTeX2: Modelo de Artigo Acadêmico em conformidade com
% ABNT NBR 6022:2003: Informação e documentação - Artigo em publicação 
% periódica científica impressa - Apresentação
% ------------------------------------------------------------------------
% ------------------------------------------------------------------------

\documentclass[
	% -- opções da classe memoir --
	article,			% indica que é um artigo acadêmico
	11pt,				% tamanho da fonte
	oneside,			% para impressão apenas no verso. Oposto a twoside
	a4paper,			% tamanho do papel. 
	% -- opções da classe abntex2 --
	%chapter=TITLE,		% títulos de capítulos convertidos em letras maiúsculas
	%section=TITLE,		% títulos de seções convertidos em letras maiúsculas
	%subsection=TITLE,	% títulos de subseções convertidos em letras maiúsculas
	%subsubsection=TITLE % títulos de subsubseções convertidos em letras maiúsculas
	% -- opções do pacote babel --
	english,			% idioma adicional para hifenização
	brazil,				% o último idioma é o principal do documento
	sumario=tradicional
	]{abntex2}


% ---
% PACOTES
% ---

% ---
% Pacotes fundamentais 
% ---
\usepackage{lmodern}			% Usa a fonte Latin Modern
\usepackage[T1]{fontenc}		% Selecao de codigos de fonte.
\usepackage[utf8]{inputenc}		% Codificacao do documento (conversão automática dos acentos)
\usepackage{indentfirst}		% Indenta o primeiro parágrafo de cada seção.
\usepackage{nomencl} 			% Lista de simbolos
\usepackage{color}				% Controle das cores
\usepackage{graphicx}			% Inclusão de gráficos
\usepackage{microtype} 			% para melhorias de justificação
% ---
		
% ---
% Pacotes adicionais, usados apenas no âmbito do Modelo Canônico do abnteX2
% ---
\usepackage{lipsum}				% para geração de dummy text
% ---
		
% ---
% Pacotes de citações
% ---
\usepackage[brazilian,hyperpageref]{backref}	 % Paginas com as citações na bibl
\usepackage[alf,num]{abntex2cite}	% Citações padrão ABNT
% ---

\usepackage{amsmath}

% ---
% Configurações do pacote backref
% Usado sem a opção hyperpageref de backref
\renewcommand{\backrefpagesname}{Citado na(s) página(s):~}
% Texto padrão antes do número das páginas
\renewcommand{\backref}{}
% Define os textos da citação
\renewcommand*{\backrefalt}[4]{
	\ifcase #1 %
		Nenhuma citação no texto.%
	\or
		Citado na página #2.%
	\else
		Citado #1 vezes nas páginas #2.%
	\fi}%
% ---

% ---
% Informações de dados para CAPA e FOLHA DE ROSTO
% ---
\titulo{Uma simulação de Burnout}
\autor{Danilo Fonte}
\local{Brasil}
\data{Abril, 2025}
% ---

% ---
% Configurações de aparência do PDF final

% alterando o aspecto da cor azul
\definecolor{blue}{RGB}{41,5,195}

% informações do PDF
\makeatletter
\hypersetup{
     	%pagebackref=true,
		pdftitle={\@title}, 
		pdfauthor={\@author},
    	pdfsubject={Modelo de artigo científico com abnTeX2},
	    pdfcreator={LaTeX with abnTeX2},
		pdfkeywords={abnt}{latex}{abntex}{abntex2}{atigo científico}, 
		colorlinks=true,       		% false: boxed links; true: colored links
    	linkcolor=blue,          	% color of internal links
    	citecolor=blue,        		% color of links to bibliography
    	filecolor=magenta,      		% color of file links
		urlcolor=blue,
		bookmarksdepth=4
}
\makeatother
% --- 

% ---
% compila o indice
% ---
\makeindex
% ---

% ---
% Altera as margens padrões
% ---
\setlrmarginsandblock{3cm}{3cm}{*}
\setulmarginsandblock{3cm}{3cm}{*}
\checkandfixthelayout
% ---

% --- 
% Espaçamentos entre linhas e parágrafos 
% --- 

% O tamanho do parágrafo é dado por:
\setlength{\parindent}{1.3cm}

% Controle do espaçamento entre um parágrafo e outro:
\setlength{\parskip}{0.2cm}  % tente também \onelineskip

% Espaçamento simples
\SingleSpacing

% ----
% Início do documento
% ----
\begin{document}

% Retira espaço extra obsoleto entre as frases.
\frenchspacing 

\maketitle

\section*{Introdução}
\addcontentsline{toc}{section}{Introdução}

A síndrome de Burnout é um fenômeno associado à exaustão mental resultante de ambientes corporativos estressantes e exigentes. A Organização Mundial da Saúde (OMS) define o Burnout como um estado de estresse crônico relacionado ao trabalho, caracterizado por exaustão, cinismo e redução da eficácia profissional \cite{Downey2023}. Além disso, o acúmulo de expectativas não atendidas e o gerenciamento inadequado do estresse contribuem significativamente para o desenvolvimento dessa condição.
Em contextos corporativos modernos, o gerenciamento eficaz dos ciclos de estresse e recompensa dos funcionários é essencial para promover o bem-estar e o equilíbrio no ambiente de trabalho. A satisfação e a saúde mental dos colaboradores impactam diretamente a produtividade e a retenção de talentos, tornando-se fatores estratégicos para as organizações  \cite{Slusarz2022}.
Para lidar com essa questão, modelos baseados na Teoria dos Jogos Evolutivos oferecem uma abordagem promissora, pois permitem analisar como recompensas e penalidades impactam os trabalhadores ao longo do tempo. Esses modelos possibilitam encontrar um ponto de equilíbrio que maximize o engajamento e a satisfação dos funcionários \cite{Sanfey2003}.
Neste contexto, este trabalho propõe o uso de Algoritmos Genéticos para otimizar o tabuleiro de um modelo previamente estabelecido \cite{Zhang2020Burnout} de interação entre empregador e 
 múltiplos empregados. O objetivo é identificar um equilíbrio sustentável que maximize o bem-estar ao longo do tempo, considerando a dinâmica de recompensas e penalidades dentro da organização \cite{Zhang2020Burnout}.

\bibliographystyle{abntex2-num}
\bibliography{referencias}

\section{Explificação do Modelo Base}
O modelo utilizado como base para esta simulação foi proposto por \cite{Zhang2020Burnout} e descreve a interação entre orientador e a universidade. Para este estudo, esses agentes serão reinterpretados como funcionário e empresa, mantendo as mesmas características do modelo original. Assume-se que o funcionário seja um agente totalmente racional, capaz de identificar estratégias que maximizem seus ganhos e minimizem suas perdas, sempre considerando as recompensas ideais com base em expectativas presentes e futuras.

\subsection{Definição do custo de um Funcionário}

O custo de um funcionário pode ser analisado a partir de três componentes principais: custo atual, custo potencial e custo futuro.

\begin{itemize}
    \item O \textbf{custo atual} inclui fatores como tempo dedicado ao trabalho, carga de trabalho objetiva e o investimento em autoaperfeiçoamento.
    \item O \textbf{custo potencial} é determinado pelo desgaste mental e pelos custos relacionados à execução do trabalho, sendo influenciado negativamente por fatores externos, como burnout e eventos adversos.
    \item O \textbf{custo futuro} está associado à incerteza da carreira e às ações necessárias para o planejamento profissional do funcionário.
\end{itemize}

A equação que representa o custo total do funcionário é dada por:

\begin{equation}
    G = \beta_1 g_1 + \beta_2 g_2 + \beta_3 g_3
\end{equation}

\begin{equation}
    G = \beta_1 (s t + z - l) + \beta_2 (h^2 + c) + \beta_3 h^2
\end{equation}

Sujeito às seguintes condições:

\begin{equation}
    \beta_1, \beta_2, \beta_3 > 0, \quad \beta_1 + \beta_2 + \beta_3 = 1
\end{equation}

\begin{equation}
    0 < s, t, z, l, h, c < 1, \quad l < z
\end{equation}

Onde:

\begin{itemize}
    \item \( G \) representa o custo total do funcionário.
    \item \( g_1, g_2, g_3 \) correspondem, respectivamente, ao custo atual, custo potencial e custo futuro.
    \item \( s \) é o custo associado ao tempo dedicado ao trabalho.
    \item \( t \) representa a carga de trabalho objetiva.
    \item \( z \) é o custo de autoaperfeiçoamento.
    \item \( l \) mede a percepção do funcionário sobre a importância do autoaperfeiçoamento.
    \item \( h^2 \) refere-se ao custo do planejamento de carreira.
    \item \( c \) representa a possibilidade de eventos inesperados no ambiente de trabalho.
    \item Os coeficientes \( \beta_1, \beta_2, \beta_3 \) indicam a importância relativa de cada um dos custos, garantindo que sua soma seja igual a 1 (\( \beta_1 + \beta_2 + \beta_3 = 1 \)).
\end{itemize}

O modelo base também impõe restrições às variáveis, garantindo que:

\[
0 < s, t, z, l, h, c < 1, \quad l < z
\]

Isso reflete a premissa de que funcionários mais consistentes em seu desempenho e planejamento de carreira tendem a apresentar um custo total reduzido e maior eficiência no trabalho.

\subsection{Definição do ganho de um funcionário}
O ganho de um funcionário é modelado a partir de três componentes: ganhos atuais (\( f_1 \)), ganhos potenciais (\( f_2 \)) e ganhos futuros (\( f_3 \)).

\begin{itemize}
    \item \textbf{Ganhos atuais}: incluem salário (\( d \)) e status profissional externo (\( e \)).
    \item \textbf{Ganhos potenciais}: referem-se ao planejamento de carreira (\( h_1 \)), autoaperfeiçoamento (\( l \)) e avaliação de performance (\( j \)), ponderada por \( \lambda_1 \).
    \item \textbf{Ganhos futuros}: abrangem o crescimento do funcionário na empresa, considerando carga de trabalho (\( st \)), planejamento de carreira e reconhecimento profissional.
\end{itemize}

O ganho total do funcionário (\( F \)) é expresso como:

\begin{equation}
    F = \alpha_1 f_1 + \alpha_2 f_2 + \alpha_3 f_3 = \alpha_1 (2d + e) + \alpha_2 (h_1 + l + \lambda_1 j) + \alpha_3 (st + h_1 + l + j)
\end{equation}

Onde:

\begin{equation}
    \alpha_1, \alpha_2, \alpha_3 > 0, \quad \alpha_1 + \alpha_2 + \alpha_3 = 1, \quad 0 < d, e, t, h_1, l, j, s < 1
\end{equation}

As perdas e ganhos líquidos (\( M \)) são dados por:

\begin{equation}
    M = F - G = 2\alpha_1 d + \alpha_2 e + (\alpha_2 + \alpha_3) h_1 - (\beta_2 + \beta_3) h_2 + (\alpha_2 + \alpha_3 + \beta_1) l 
\end{equation}

\begin{equation}
    + (\alpha_2 \lambda_1 + \alpha_3) j + (\alpha_3 - \beta_1) st - \beta_1 z - \beta_2 c
\end{equation}

Onde \( G \) representa os custos do funcionário, considerando planejamento de carreira (\( h_2 \)), autoaperfeiçoamento (\( z \)) e eventos inesperados (\( c \)).

O modelo sugere que um alto nível de autoaperfeiçoamento, planejamento de carreira e reconhecimento profissional resulta em maior ganho e menor custo para o funcionário.


\section{Teoria dos Jogos Evolutivos}

\subsection{Equilibrio de Nash}

\section{Simulação}

\subsection{Restrições da Simulação}

\section{Resultados}

\section{Conclusões}

\end{document}
