\documentclass{article}
\usepackage{graphicx} % Required for inserting images

\title{artigo-computação-natural}
\author{danilo dos santos da fonte}
\date{April 2025}

\begin{document}

\maketitle

\section{Introdução}

A síndrome de Burnout é um fenômeno associado à exaustão mental resultante de ambientes corporativos estressantes e exigentes. A Organização Mundial da Saúde (OMS) define o Burnout como um estado de estresse crônico relacionado ao trabalho, caracterizado por exaustão, cinismo e redução da eficácia profissional \cite{Downey2023}. Além disso, o acúmulo de expectativas não atendidas e o gerenciamento inadequado do estresse contribuem significativamente para o desenvolvimento dessa condição \cite{Zhang2020Burnout}.
Em contextos corporativos modernos, o gerenciamento eficaz dos ciclos de estresse e recompensa dos funcionários é essencial para promover o bem-estar e o equilíbrio no ambiente de trabalho [referência]. A satisfação e a saúde mental dos colaboradores impactam diretamente a produtividade e a retenção de talentos, tornando-se fatores estratégicos para as organizações [referência].
Para lidar com essa questão, modelos baseados na Teoria dos Jogos Evolutivos oferecem uma abordagem promissora, pois permitem analisar como recompensas e penalidades impactam os trabalhadores ao longo do tempo. Esses modelos possibilitam encontrar um ponto de equilíbrio que maximize o engajamento e a satisfação dos funcionários \cite{Zhang2020Burnout}.
Neste contexto, este trabalho propõe o uso de Algoritmos Genéticos para otimizar um modelo previamente estabelecido de interação entre empregador e empregado. O objetivo é identificar um equilíbrio sustentável que maximize o bem-estar ao longo do tempo, considerando a dinâmica de recompensas e penalidades dentro da organização [referência ao artigo base da simulação].

\bibliographystyle{plain}
\bibliography{referencias}


\end{document}
